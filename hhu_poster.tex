%------------------------------------------------------------------------------

%   PACKAGES AND OTHER DOCUMENT CONFIGURATIONS

%------------------------------------------------------------------------------

\documentclass[final,hyperref={pdfpagelabels=false},xcolor=table]{beamer}

\usepackage[orientation=portrait,size=a0,scale=1.3]{beamerposter} % Use the beamerposter package for laying out the poster with a portrait orientation and an a0 paper size

\usetheme{HHU}

\usepackage[utf8]{inputenc} % allow utf-8 input
\usepackage{blindtext}
\usepackage{multicol}
\usepackage[utf8]{inputenc}

%\usepackage[table,xcdraw]{xcolor}
\usepackage{amsmath,amsthm,amssymb,latexsym} % For including math equations, theorems, symbols, etc
\usepackage[document]{ragged2e}
\usepackage{times}\usefonttheme{professionalfonts}  % Uncomment to use Times as the main font
\usefonttheme[onlymath]{serif} % Uncomment to use a Serif font within math environments
%\boldmath % Use bold for everything within the math environment
\usepackage{booktabs} % Top and bottom rules for tables
\usepackage{microtype}
\usepackage{tikz}
\usepackage{tipa}

\usetikzlibrary{shapes.geometric, arrows}

\usecaptiontemplate{\small\structure{\insertcaptionname~\insertcaptionnumber: }\insert  caption} % A fix for figure numbering

\newcommand{\shrink}{-15pt}

\def\imagetop#1{\vtop{\null\hbox{#1}}}

\let\oldbibliography\thebibliography
\renewcommand{\thebibliography}[1]{\oldbibliography{#1}
\setlength{\itemsep}{-10pt}}

\usepackage{pgf}
\usepackage{newunicodechar}
\usepackage{booktabs}

%-------------------------------------------------------------------------------

%   TITLE SECTION

%-------------------------------------------------------------------------------

\title{TITLE OF THE POSTER} % Poster title
\author{author1, author2, author3 etc}
\institute{Heinrich-Heine-University Düsseldorf, Germany\\\vspace{4mm}
\texttt{\{author1,author2,author3\}@hhu.de}}

%--------------------------------------------------------------------------------

%   FOOTER TEXT

%--------------------------------------------------------------------------------

\newcommand{\leftfoot}{} % Left footer text
\newcommand{\rightfoot}{} % Right footer text

%--------------------------------------------------------------------------------

\begin{document}
\addtobeamertemplate{block end}{}{\vspace*{2ex}} % White space under blocks

\begin{frame}[t] % The whole poster is enclosed in one beamer frame

\begin{columns}[t] % The whole poster consists of three major columns, each of which can be subdivided further with another \begin{columns} block - the [t] argument aligns each column's content to the top. 

\begin{column}{.02\textwidth}\end{column} % Empty spacer column

%%%%%%%%%%%%%%%%%%%%%%%%%%%%%%%%%%%

% HOW TO USE

% If you want 3 columns, use \begin{column}{.3\textwidth} so that each column is equal to 1/3 of the whole text.

% If you only want to use two columns, use \begin{column}{.47\textwidth}

% use \begin{block}{The idea} to create a new topic block

%%%%%%%%%%%%%%%%%%%%%%%%%%%%%%%%%%%%%%%%%%

%% Column 1

%%%%%%%%%%%%%%%%%%%%%%%%%%%%%%%%%%%%%%%%%%
  \begin{column}{.3\textwidth} % start of the 1st column

    \vspace{\shrink}
    \begin{block}{The Problem}
    
    
    
    
    \end{block}

      \begin{block}{The idea}
     \begin{itemize}
          \item idea1
          \item idea2
          \item idea3
          \item idea4
      \end{itemize}
    \end{block}
    
    \begin{block}{The method}
     
    What did you do?
    
    \end{block}

  \end{column} % End of the 1st column
\begin{column}{.02\textwidth}\end{column} % Empty space column so the bars dont stick to each other
%%%%%%%%%%%%%%%%%%%%%%%%%%%%%%%%%%%%%%%%%%

%% Column 2

%%%%%%%%%%%%%%%%%%%%%%%%%%%%%%%%%%%%%%%%%%
  
  \begin{column}{.3\textwidth} % start of 2nd column
    \vspace{\shrink}
    
          \begin{block}{The data}
      \setlength{\tabcolsep}{0.5em} % for the horizontal padding
    {\renewcommand{\arraystretch}{1.2}% for the vertical padding
  \begin{tabular}{|c|p{15cm}<{\centering}|}
  \hline
    Variable1 & Variable2\\
    \hline
    value & Value \\
    \hline
    value & Value \\
    \hline
    value & Value\\
    \hline
    value & Value \\
    \hline
\end{tabular}
}
     
    \end{block}
    
    \begin{block}{Results}
    Your findings
     
    \end{block}
      

  \end{column} % End of the 2nd column
\begin{column}{.02\textwidth}\end{column} % Empty spacer column

%%%%%%%%%%%%%%%%%%%%%%%%%%%%%%%%%%%%%%%%%%
%% Column 3
%%%%%%%%%%%%%%%%%%%%%%%%%%%%%%%%%%%%%%%%%%  
  \begin{column}{.3\textwidth} % 3rd column

    \begin{block}{Discussion}
     \begin{itemize}
          \item No
          \item Yes
     \end{itemize}

    \end{block}
    \begin{block}{References}
     % \nocite{*} % Insert publications even if they are not cited in the poster
      \linespread{0.928}\selectfont
      \footnotesize{\bibliographystyle{unsrt}
      \bibliography{hhu_poster}}
    \end{block}

  \end{column} % End of the 3rd column
  \begin{column}{.02\textwidth}\end{column} % Empty spacer column
\end{columns} % End of all the columns in the poster
\end{frame} % End of the enclosing frame
\end{document}
